\documentclass[12pt]{article}
\usepackage{hyperref}

\begin{document}
\title{Modelling Low Earth Orbit Constellations for Networking}
\author{Joseph McGuchan}
\maketitle
\thispagestyle{empty}

\section{Introduction}

SpaceX are planning to launch a constellation of 4,425 low Earth orbit communication satellites in the next few years. The objective of this constellation, called Starlink, is to provide low-latency internet connection across the world.

The satellites in this network will be in constant motion, not just relative to the ground, but relative to one another, creating a network with a constantly changing topology and associated latencies. The question of how to route signals across such a network has not been thoroughly explored, but it will become increasingly relevant as more and more companies build similar constellations.

My goals are:
\begin{enumerate}
	\item To develop multiple routing algorithms for this network.
	\item To compare these algorithms on latency, fault tolerance, and response to high demand.
	\item To create visualisations of these routing algorithms at work.
\end{enumerate}

This network, and networks like it, are the future of the internet.

%TODO

\section{What is Starlink?}

%TODO

As of 21/11/18, there are two companies offering sattelite internet services, Excede\cite{ExcedeWebsite}, and Hughes, whose 9202 BGAN Land Portable Satellite Terminal offers connection speeds up to 464kbps\cite{HughesWebsite}. These companies largely target domestic use in rural areas which don’t have a faster coverage, and corporations, providing internet connections to airplanes and cargo ships. Currently, Sattelite Internet connection is a last resort, something turned to when conventional means of connection are not available, Starlink intends to invert this, turning sattelite internet into the premium option, 

%TODO

SpaceX have already sent up two test sattelites, and according the Elon Musk, they are working very well, providing a latency of only 25ms\cite{ElonMuskTweet}.

%TODO

\subsection{The Structure of Starlink}

There is a lot we do not know about Starlink, but this is what we can infer from SpaceX’s application to the FCC\cite{FCCApplication}, and their technical attachment\cite{TechnicalAttachment}.




\begin{thebibliography}{9}
\bibitem{ExedeWebsite} \url{https://www.exede.com}
\bibitem{HughesWebsite} \url{https://www.hughes.com}
\bibitem{ElonMuskTweet} \url{https://twitter.com/elonmusk/status/1000453321121923072}
\bibitem{FCCApplication} \url{licensing.fcc.gov/cgi-bin/ws.exe/prod/ib/forms/reports/related_filing.hts?f_key=-289550&f_number=SATLOA2016111500118}
\bibitem{TechnicalAttachment} \url{https://licensing.fcc.gov/myibfs/download.do?attachment_key=1158350}
\end{thebibliography}

\end{document}